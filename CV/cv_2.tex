%%%%%%%%%%%%%%%%%%%%%%%%%%%%%%%%%%%%%%%%%
% Medium Length Graduate Curriculum Vitae
% LaTeX Template
% Version 1.1 (9/12/12)
%
% This template has been downloaded from:
% http://www.LaTeXTemplates.com
%
% Original author:
% Rensselaer Polytechnic Institute (http://www.rpi.edu/dept/arc/training/latex/resumes/)
%
% Important note:
% This template requires the res.cls file to be in the same directory as the
% .tex file. The res.cls file provides the resume style used for structuring the
% document.
%
%%%%%%%%%%%%%%%%%%%%%%%%%%%%%%%%%%%%%%%%%

%----------------------------------------------------------------------------------------
%	PACKAGES AND OTHER DOCUMENT CONFIGURATIONS
%----------------------------------------------------------------------------------------

\documentclass[margin]{res} % Use the res.cls style, the font size can be changed to 11pt or 12pt here

\usepackage{hyperref}
\usepackage{helvet} % Default font is the helvetica postscript font
%\usepackage{newcent} % To change the default font to the new century schoolbook postscript font uncomment this line and comment the one above


\setlength{\textwidth}{5.1in} % Text width of the document

\begin{document}

%----------------------------------------------------------------------------------------
%	NAME AND ADDRESS SECTION
%----------------------------------------------------------------------------------------

\moveleft.5\hoffset\centerline{\large\bf Andrea Petri} % Your name at the top
 
\moveleft\hoffset\vbox{\hrule width\resumewidth height 1pt}\smallskip % Horizontal line after name; adjust line thickness by changing the '1pt'
 
\moveleft.5\hoffset\centerline{538 West 120th Street} % Your address
\moveleft.5\hoffset\centerline{New York, NY 10027}
\moveleft.5\hoffset\centerline{apetri@phys.columbia.edu}
\moveleft.5\hoffset\centerline{\url{http://apetri.me}}

%----------------------------------------------------------------------------------------

\begin{resume}

%----------------------------------------------------------------------------------------
%	EDUCATION SECTION
%----------------------------------------------------------------------------------------

\section{EDUCATION}

{\sl Laurea Specialistica,} Theoretical Physics, June 2011 \\
Scuola Normale Superiore, Pisa, Italy \\ 
Thesis advisor: Prof. Andrea Ferrara \\ \\ 
{\sl Master of Arts,} Physics, May 2013 \\ 
Columbia University \\ \\ 
{\sl Master of Philosophy,} Physics, May 2014 \\ 
Columbia University \\ \\ 
{\sl Doctor of Philosophy,} Physics, Expected June 2017 \\ 
Columbia University \\
Research advisors: Prof. Zolt\`an Haiman, Prof. Morgan May
 
%----------------------------------------------------------------------------------------
%	POSITIONS SECTION
%----------------------------------------------------------------------------------------
 
\section{POSITIONS}

{\sl Graduate student researcher, Columbia University Physics Department} \hfill 2011- \\

%----------------------------------------------------------------------------------------
%	PROFESSIONAL ACTIVITIES SECTION
%---------------------------------------------------------------------------------------- 

\section{PROFESSIONAL \\ ACTIVITIES} 

Served as peer reviewer for the American Astronomical Society (AAS) and for the MNRAS journal
%----------------------------------------------------------------------------------------
%	TEACHING SECTION
%----------------------------------------------------------------------------------------

\section{TEACHING} 
{\sl Graduate student instructor} \hfill 2011-\\ 
Columbia University, NY, Physics Department 
\begin{itemize}
\item Introductory Physics Lab (pre-medical) Fall 2011, Spring 2012
\item Introductory Physics Lab (engineers) Fall 2012, Spring 2013
\item Physical Cosmology (TA, grading), Fall 2012
\item Particle Astrophysics and Cosmology (TA, recitations), Spring 2013 
\item EKA Advanced Physics Laboratory (TA), Fall 2013-
\item Particle Astrophysics and Cosmology (TA, grading), Spring 2015
\item Particle Astrophysics and Cosmology (TA, recitations, homework solutions writeup), Spring 2016
\end{itemize}

{\sl Instructor} \hfill 2011- \\
Columbia University, NY, Science Honors Program\\
Introduction to Modern Cosmology for high school students
%----------------------------------------------------------------------------------------
%      PUBLICATIONS
%----------------------------------------------------------------------------------------

\section{PUBLICATIONS}

\subsection{First authored}

\begin{enumerate}

\item{\sl Cosmology with photometric weak lensing surveys: constraints with redshift tomography of convergence peaks and moments} \\
A. Petri, M. May, Z. Haiman, arXiv:1605.01100, submitted to PRD (under peer review)
\item {\sl Mocking the Weak Lensing universe: the LensTools python computing package} \\
A.Petri; Astronomy \& Computing, Elsevier, \textbf{17}, 73-79 (2016)
\item {\sl Sample variance in weak lensing: how many simulations are required?} \\
A.Petri, Z.Haiman, M.May; Phys. Rev. D. \textbf{93}, 063524 (2016)
\item {\sl Emulating the CFHTLenS weak lensing data: Cosmological constraints from moments and Minkowski functionals} \\
A.Petri, J. Liu, Z.Haiman, M.May, L.Hui, J.M.Kratochvil; Phys. Rev. D. \textbf{91}, 103511 (2015)
\item {\sl Impact of spurious shear on cosmological parameter estimates from weak lensing observables} \\
A.Petri, M.May, Z.Haiman, J.M.Kratochvil; Phys. Rev. D. \textbf{90}, 123015 (2014)
\item {\sl Cosmology with Minkowski Functionals and moments of the weak lensing convergence field} \\
A.Petri, Z.Haiman, L.Hui, M.May, J.M.Kratochvil; Phys. Rev. D. \textbf{88}, 123002 (2013)
\item {\sl Supermassive black hole ancestors} \\
A.Petri, A.Ferrara, R.Salvaterra; Mon. Not. R. Astron. Soc. \textbf{422}, 1690-1699 (2012)

\end{enumerate}

\subsection{Collaborations}

\begin{itemize}

\item {\sl CMB Lensing Beyond the Power Spectrum: Cosmological Constraints from the One-Point PDF and Peak Counts}\\
J. Liu, J. Coin Hill, B. D. Sherwin, A. Petri, V. Bohm, Z. Haiman, submitted to PRD (under peer review)
\item {\sl Consequences of CCD imperfections for cosmology determined by weak lensing surveys: From laboratory measurements to cosmological parameter bias} \\ 
Y.Okura, A.Petri, M.May, A.Plazas, T.Tamagawa; Astrophys. Journal, 825-1, \textbf{61} (2016)
\item {\sl Cosmology constraints from the weak lensing peak counts and the power spectrum in CFHTLenS data} \\ 
J.Liu, A.Petri, Z.Haiman, L.Hui, J.M.Kratochvil, M.May; Phys. Rev D. \textbf{91}, 063507 (2015)
\end{itemize}

%----------------------------------------------------------------------------------------
%	AWARDS SECTION
%----------------------------------------------------------------------------------------

\section{AWARDS}
\begin{itemize}
\item Co-recipient of the Allan M. Sachs Teaching Award for contributions to the educational programs in the Columbia University Physics Department (May 2016)
\item Bronze medalist, 37th International Physics Olympiad, Singapore (July 2006) 
\end{itemize}

%----------------------------------------------------------------------------------------
%	TALKS SECTION
%----------------------------------------------------------------------------------------

\section{INVITED TALKS}

\section{CONFERENCE TALKS}
\begin{itemize}

\item {\sl  Non--gaussian statistics task force: update}\\
LSST DESC collaboration meeting, Stanford Linear Accelerator Center, Stanford, March 9th 2016

\item {\sl  Non--gaussian statistics task force: the simulations}\\
LSST DESC collaboration meeting, Argonne National Laboratory, Chicago, October 28th 2015

\item {\sl Constraining cosmology with weak lensing: higher moments in CFHT}\\
AstroFest 2015, Columbia University, September 11th 2015

\item {\sl Cosmology with Minkowski Functionals and Moments of Weak Lensing Fields}\\
Santa Fe Cosmology Workshop, July 2014

\item{\sl Cosmology with Minkowski Functionals and moments of the weak lensing convergence field}\\
27th Symposium on Relativistic Astrophysics, December 8-13, 2013, Dallas, TX \\
\url{http://nsm.utdallas.edu/texas2013/proceedings/4/4/b/Petri.pdf}


\end{itemize}


\end{resume}
\end{document}