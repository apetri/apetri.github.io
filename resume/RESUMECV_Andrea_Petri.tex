%%%%%%%%%%%%%%%%%%%%%%%%%%%%%%%%%%%%%%%%%
% Medium Length Professional CV
% LaTeX Template
% Version 2.0 (8/5/13)
%
% This template has been downloaded from:
% http://www.LaTeXTemplates.com
%
% Original author:
% Trey Hunner (http://www.treyhunner.com/)
%
% Important note:
% This template requires the resume.cls file to be in the same directory as the
% .tex file. The resume.cls file provides the resume style used for structuring the
% document.
%
%%%%%%%%%%%%%%%%%%%%%%%%%%%%%%%%%%%%%%%%%

%----------------------------------------------------------------------------------------
%	PACKAGES AND OTHER DOCUMENT CONFIGURATIONS
%----------------------------------------------------------------------------------------

\documentclass{resume} % Use the custom resume.cls style

\usepackage[left=0.75in,top=0.6in,right=0.75in,bottom=0.6in]{geometry} % Document margins
\usepackage{mathptmx}
\usepackage{hyperref}

\name{Andrea Petri} % Your name
\address{932 Pupin Hall \\ 538 West 120th Street \\ New York, NY 10027} % Your address
\address{http://apetri.me}
\address{(917)~$\cdot$~969~$\cdot$~7212 \\ apetri@phys.columbia.edu} % Your phone number and email



\begin{document}

%----------------------------------------------------------------------------------------
%	EDUCATION SECTION
%----------------------------------------------------------------------------------------

\begin{rSection}{Education}

{\bf Columbia University, Graduate School of Arts and Sciences} \hfill {\em August 2011 - present} \\ 
PhD. Physics \hfill {\em expected 2017} \\ M.A. Physics \hfill {\em June 2013} 
\\
\textit{Relevant coursework:} \\
Advanced Programming \hspace{4pt} Statistical Mechanics \hspace{4pt} Quantum Mechanics\\
Physical Cosmology \hspace{4pt} Classical Fields and Waves \hspace{4pt} Quantum Field Theory

{\bf Scuola Normale Superiore, Pisa, Italy} \hfill {\em July 2011} \\ 
B.A. in Physics \\

\end{rSection}

%----------------------------------------------------------------------------------------
%	WORK EXPERIENCE SECTION
%----------------------------------------------------------------------------------------

\begin{rSection}{Experience}

\begin{rSubsection}{Morgan Stanley - Institutional Equity Division}{June 2015 - August 2015}{Electronic Market Making desk}{New York}
\item Analyzed stock market historical data, with particular focus on US equity market trades from 2009 to 2014
\item Developed mathematical models and algorithms for intra--day volume forecasts 
\end{rSubsection}

\begin{rSubsection}{Project experience}{Fall 2013 - Present}{Software engineering}{Columbia University, NY}
\item Developed the LensTools Python library, that will prove useful in Weak Gravitational Lensing data analyses, with particular focus on ray-tracing simulations, astrophysical image analysis, data reduction and statistical inferences of model parameters from observations (project URL \url{http://lenstools.rtfd.org})
\item Implemented from scratch the client and server side components of a three tier simple database service, using the C language socket API (code repository available on request)
\end{rSubsection}

%------------------------------------------------

\begin{rSubsection}{Research}{Summer 2012 - Present}{Astrophysics -- Large Scale Structure of the Universe}{Columbia University, NY}
\item Worked on Cosmic Microwawe Background (CMB) data analysis, with particular focus on temperature image reconstruction starting from raw time ordered data (bolometric and pointing) 
\item Contributed to the development of CMB map-making software, implemented the corrections for pointing and calibration offsets
\item Handled several supercomputing tasks, including planning and production of a 30TB simulated dataset featuring Cosmological N-body systems
\item Conducted statistical analysis of Cosmological Large Scale Structure simulated images, with particular emphasis on the development and implementation of new techniques to constrain physical model parameters
\item Served as peer reviewer for the journal Monthly Notices of the Royal Astronomical Society
\end{rSubsection}

%------------------------------------------------

\begin{rSubsection}{Teaching}{Fall 2012 - Present}{Graduate student instructor}{Columbia University, NY}
\item Taught several Physics Laboratory introductory courses aimed at pre-medical and engineering track students
\item Designed and taught as co-instructor a Modern Cosmology class aimed at high school students in the Columbia Science Honors Program (SHP)   
\end{rSubsection}

\end{rSection}

%----------------------------------------------------------------------------------------
%	TECHNICAL STRENGTHS SECTION
%----------------------------------------------------------------------------------------

\begin{rSection}{Technical Strengths}

\begin{tabular}{ @{} >{\bfseries}l @{\hspace{6ex}} l }
Mathematical tools & Linear algebra, bayesian statistics, image processing \\
Programming Languages & Python, C/C++, Fortran90, Bash, R \\
Protocols \& APIs & Object Oriented Programming, Parallel Computing (MPI), TCP/IP sockets, HTTP \\
Databases & MySQL \\
Tools & Distributed source control (git, mercurial) 
\end{tabular}

\end{rSection}

%----------------------------------------------------------------------------------------
%	EXAMPLE SECTION
%----------------------------------------------------------------------------------------

%\begin{rSection}{Section Name}

%Section content\ldots

%\end{rSection}

%----------------------------------------------------------------------------------------

\end{document}
